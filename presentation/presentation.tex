\documentclass{beamer}
\usepackage{bm}
\usetheme{Darmstadt}
\title{Dissipation and dispersion in finite difference solutions of hyperbolic PDEs}
\subtitle{Numerical solutions of PDEs}
\author{Gabriele Cimador}
\institute{Data Science and Scientific computing at Università di Trieste}
\date{\today}
\begin{document}
\begin{frame}
\titlepage
\end{frame}
\begin{frame}
\frametitle{Summary}
\tableofcontents
\end{frame}
\section{Hyperbolic problems}
\subsection{What is an hyperbolic problem?}
\begin{frame}
Usually transport wave-like phenomena with finite speed of propagation. Examples:
\begin{itemize}
  \item 1d transport/advection eq. $u_t + a u_x = 0$
  \item Conservation laws $u_t + (f(u))_x = 0$
  \item Wave eq. $u_{tt} + cu_{xx} = 0$
\end{itemize}
\end{frame}
\begin{frame}
All of the prevoius can be grouped with the general tranport system of equations:
\begin{align*}
\bm{u}_t + A\bm{u}_x = 0
\end{align*}
where $A$ matrix which can depend on $t, x, \bm{u}$ and has a full set of real eigenvalues.
\\ e.g. for the conservation law equation
\begin{align*}
u_t + (f(u))_x = 0 \Rightarrow u_t + A(u)u_x = 0
\end{align*}
 where $A(u) = \frac{\partial f}{\partial u}$.
\end{frame}
\subsection{Some comments}
\begin{frame}
\begin{itemize}
  \item There is no dissipation: $\lVert u(t,\cdot) \rVert_{L_2} = \lVert u_0\rVert_{L_2}$
  \item Information propagates at finite speed
  \item Discontinuites in initial data is propagated $\Rightarrow$ discontinue solution
  \item CFL condition necessary for convergence of a finite difference scheme
\end{itemize}
\end{frame}
\section{Finite differences}
\begin{frame}
\frametitle{Finite differences}
If the PDE is defined in a domain $I \times \Omega$ where $I$ is the time interval $[0, T_f]$ and $\Omega$ is the space domain of one variable $[a,b]$, we can discretize the PDE domain with $N_t \times N_x$ points. We can than define a general explicit difference scheme at time $t = t_n$ as
\begin{align*}
v_j^{n+1} = \sum_{i = -l}^r \beta_i v_{j+i}^n
\end{align*}
where $j = l, ..., N_x - r -1$, $n = 0, ..., N_t$ and $v_j^n$ is a mesh function over the discretised domain.
\end{frame}
\begin{frame}
If: \begin{itemize}
\item PDE has constant coefficients
\item Problem is defined on infinte mesh or has periodic boundary conditions
\end{itemize}
Can perform a Fourier analysis of how the difference scheme acts on the initial condition. \\ If $\hat{v}(t,\xi)$ is Fourier transform of the solution, we have $v(t,x) = \frac{1}{\sqrt{2\pi}}\int_{-\infty}^{\infty}e^{\iota\xi x}\hat{v}(t,\xi)d\xi$ \\
\end{frame}
\begin{frame}
The difference scheme gives
\begin{align*}
\int_{-\infty}^{\infty}e^{\iota\xi x}\hat{v}(t + \tau,\xi)d\xi = & \sum_{i=-l}^{r}\beta_i\int_{-\infty}^{\infty}e^{\iota\xi (x + ih)}\hat{v}(t,\xi)d\xi \\
= & \int_{-\infty}^{\infty} \sum_{i=-l}^{r}\beta_i e^{\iota\xi (x + ih)}\hat{v}(t,\xi)d\xi \\
\end{align*}
\begin{align*}
\Rightarrow \hat{v}(t + \tau, \xi) = g(\zeta)\hat{v}(t,\xi), \\
\zeta = \xi h, \ \tau = \frac{T_f}{N_t}, \ h = \frac{b - a}{N_x}
\end{align*}
\end{frame}
\begin{frame}
the amplification factor
\end{frame}
\section{Advection equation}
\begin{frame}
\frametitle{Advection equation}
We consider the advection equation equipped with initial and boundary condition:
$$
\begin{cases}
u_t + a u_x = 0, & x,t \in [a,b] \times [0,T_f] \\
u(0,x) = u_0, & x \in [a,b]
\end{cases}
$$
The boundary condition depends on the sign of $a$. e.g. if $a > 0$ the boundary condition reads $u(a,t) = f(t) \ \forall \ t \ \in [0,T_f]$
\end{frame}
\section{Upwind method}
\begin{frame}
\frametitle{Upwind method}
If $U_j^n$ is a mesh function approximating $u$ solution of PDE over a discretisation of $[a,b] \times [0,T_f]$ with space step $h$ and time step $\tau$, then:
$$
U_j^{n+1} =
\begin{cases}
(1 - \nu) U_j^n + \nu U_{j - 1}^n & \text{if } a > 0 \\
(1 + \nu) U_j^n - \nu U_{j + 1}^n & \text{if } a < 0
\end{cases}
$$
where $\nu = a \frac{\tau}{h}, n = 0, ..., N_t - 1$
\end{frame}
\begin{frame}
tau
\end{frame}
\end{document}